\documentclass[12pt]{article}

% MATH
\usepackage{mathtools, amssymb, gensymb}
\allowdisplaybreaks
\newcommand{\dif}{\mathop{}\!\mathrm{d}}
\newcommand{\diint}{{\displaystyle \iint}}
\newcommand{\diiint}{{\displaystyle \iiint}}
\newcommand{\pd}[3][]{\frac{\partial^{#1} {#2}}{\partial {#3}^{#1}}}
\newcommand{\dpd}[3][]{\dfrac{\partial^{#1} {#2}}{\partial {#3}^{#1}}}
\newcommand{\od}[3][]{\dfrac{\dif^{#1} {#2}}{\dif {#3}^{#1}}}
\newcommand{\dod}[3][]{\dfrac{\dif^{#1} {#2}}{\dif {#3}^{#1}}}

% FLOATS
\usepackage[section]{placeins}
\usepackage{multirow}
\usepackage[margin = 2 cm]{caption}
\usepackage{subcaption}

% SPACING
\usepackage[margin=1in]{geometry} % 1 inch margins
\renewcommand{\arraystretch}{1.5} % Make LaTeX's tables look nicer


\begin{document}

\title{Bootstrapping Advanced Track AT1}
\author{TODO: Author Name}
\date{September 27, 2019}

\maketitle


% IF YOU NEED HELP WITH LaTeX:
% Here is a good resource: https://en.wikibooks.org/wiki/LaTeX



Here is a sample inline equation (\ref{eq:limerick}). Please comment out this line and equation before submitting.
\begin{align} \label{eq:limerick}
\frac{12 + 144 + 20 + 3\sqrt{4}}{7} + 5 \times 11 &= 9^2 + 0 \\
81 &= 81
\end{align}
% Note: If you do not leave a blank line after an equation, it will not indent the next line. Moral: Only leave blank lines in the body of the document if you want to start a new paragraph.



\section{Recreated Formula 1}

% TODO


\section{Recreated Formula 2}

% TODO




\section*{Questions}
% Please ask at least three questions about LaTeX and this assignment.


\begin{enumerate}
  \item TODO

  \item TODO

  \item TODO

  % Feel free to ask more questions

\end{enumerate}


\end{document}
